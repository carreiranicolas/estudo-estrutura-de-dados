\documentclass{report}

% Pacotes para acentuação e formatação

\usepackage[utf8]{inputenc}
\usepackage[T1]{fontenc}
\usepackage[brazil]{babel}
\usepackage{setspace}    % Para espaçamento
\usepackage{lipsum}      % Texto de exemplo (remova se não precisar)
\usepackage{graphicx}

\begin{document}
	
	\begin{titlepage}
		
		\centering
		\vspace*{5cm} % Espaço do topo
		
		{\Huge\bfseries Estrutura de Dados I\par} % Título
		
		\vspace{0.5cm}
		{\Large 2025/2\par} % Ano
		
		\vfill
		{\large Nicolas Ramos Carreira\par} % Nome
		
		\vspace*{2cm}
	\end{titlepage}
	
	\tableofcontents
	\newpage
	
	\chapter{Intuito}
	
	O intuito deste documento é documentar o meu aprendizado da disciplina de estrutura de dados 1. Nesta disciplina começamos estudando sobre a linguagem C até entrar nas principais estruturas de dados. 
	
	\chapter{Fundamentos em C}
	\section{Sobre a linguagem C}
	\section{Estrutura de um programa em C}
	\section{Aspectos da linguagem C}
	\subsection{Variaveis}
	\subsubsection{O que são e pra que são usadas}
	\subsubsection{Declaração de variaveis em C}
	\subsection{Tipos de dados}
	\subsubsection{Char}
	\subsubsection{Int}
	\subsubsection{Float}
	\subsubsection{Double}
	\subsubsection{Outros tipos}
	\subsection{Input e output}
	\subsubsection{Especificadores de formato}
	\subsubsection{Saída com printf()}
	\subsubsection{Uso do escape no printf()}
	\subsubsection{Entrada com scanf()}
	\subsection{Contantes}
	\subsection{Operadores}
	\subsubsection{Operadores aritméticos}
	\subsubsection{Operadores relacionais}
	\subsubsection{Operadores lógicos}
	\subsection{Coerção de tipos}
	\subsection{Condicionais}
	\subsubsection{If-else}
	\subsubsection{Swicth-case}
	\subsection{Loops}
	\subsection{Arrays}
	\subsection{Struct - Criação de tipos}
	

\end{document}

